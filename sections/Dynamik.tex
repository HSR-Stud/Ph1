\newpage
\section{Dynamik}
\subsection{Newtonsche Gesetze}
\subsubsection{Erstes Newtonsches Gesetz Trägheitsgesetz}
Ein Körper verharrt im Zustand der Ruhe oder der gleichförmigen Bewegung, wenn er
nicht durch einwirkende Kräfte gezwungen wird, seinen Zustand zu ändern. Die Gesamt-
summe der Kräfte in einem abgeschlossenen System ist unveränderlich:

\subsubsection{Zweites Newtonsches Gesetz (Aktionsgesetz)}
Die Beschleunigung eines Körpers ist umgekehrt proportional zu seiner Masse und direkt
proportional zur Kraft, die auf ihn wirkt.

\subsubsection{Drittes Newtonsches Gesetz (Actio = Reactio)}
Wirkt ein Körper A auf einen Körper B mit der Kraft ~F AB , so wirkt der Körper B mit der
entgegengesetzt gerichteten, gleich grossen Kraft $\overrightarrow{F}_{BA}$.

\subsubsection{Allgemeines Vorgehen beim Lösen von Bewegungsproblemen}
\begin{enumerate}
	\item Zeichnung anfertigen
	\item Für jeden Körper, der untersucht werden soll, wird ein Kräftediagramm eingezeichnet
	\item Ein geeignetes Koordinatensystem einführen
	\item  Das entstandene Gleichungssystem aufl"osen
	\item  Ergebnisse mit gesundem Menschenverstand aufl"osen
\end{enumerate}

$F_{G} = Gewichtskraft$
$F_{R} = Reibungskraft$
$F_{N} = Normalkraft$
$\mu_{G} = Gleitreibung$
$\mu_{H} = Haftreibung$
$W_{B} = Beschleunigungsarbeit$
$E_{kin} = Kinetische Energie$

\subsubsection{Spezielle Kräfte, Masse, Dichte, Reibung}
$F_{G} = mg$ \\
$F_{R} = \mu_{G}F_{N}$ \\
$\rho = \frac{m}{V}$ \\
$\mu = - \frac{a}{g}$ \\
$\mu_{H} = \tan{}$ \\
$F_{H_{max}} = \mu_{H}F_{H}$ \\ 

\subsubsection{Arbeit und Energie, Energieerhaltung}
Energie ist die Fähigkeit Arbeit zu leisten. Arbeit = überwinden eines Widerstandes \\
Energie kann weder verschwinden, noch aus dem Nichts entstehen. 
Wenn gelegentlich davon gesprochen wird, dass Energie 
”vernichtet“ werde, so ist damit gemeint, dass mechanische Energie 
(potentielle oder kinetische Energie) durch Reibungsarbeit in 
”Wärme“, genauer gesagt, in innere Energie, umgewandelt wird. \\
Die Gesamtenergie eines abgeschlossenen Systems ist 
unveränderlich. \\
$dW = \overrightarrow{F} \cdot \overrightarrow{ds}$
$W = Pt$
$F = \frac{P}{v}$

\subsubsection{Leistung}

$P = \frac{\Delta W}{\Delta t} = \frac{\overrightarrow{F} \Delta \overrightarrow{s}}{ \Delta t} = \overrightarrow{F} \cdot \frac{\Delta \overrightarrow{s}}{\Delta t} = \overrightarrow{F} \cdot \overrightarrow{v} = \overrightarrow{M} \cdot \omega$ \\
Umwandlung von $1 kWh = 3.6 \cdot 10^3 J $ (Überprüfen) \\
1 PS = 735.5 W


\subsubsection{Hubarbeit, Potentielle Energie}

Potentielle Energie $E_{pot} = m \cdot g \cdot h$ \\
$W_{H} \cdot \overrightarrow{F} \cdot \overrightarrow{h}$ \\
Hubarbeit: $W_{H} = E_{pot}$

\subsubsection{Spannarbeit, Spannenergie}
Spannenergie: $E_{s} = \frac{c \cdot x^2}{2}$  entspricht Spannarbeit: $W_{s} = \overrightarrow{F_{s}} \cdot \overrightarrow{x}$ \\
x: = Spannweg [m], $F_{s}$ = Spannkraft


\subsubsection{Beschleunigungsarbeit, Kinetische Energie}
$E_{kin} = \frac{mv^2}{2}$
$W_{B} = \frac{m(\Delta v)^2}{2}$

\subsubsection{Rotationsenergie}
$E_{rot} = \frac{1}{2}Jw^2$

\subsubsection{Reibungsarbeit}
$W_{R} = F_{RS}$

\subsubsection{Verformungsarbeit}
$Inelatisch: W_{D} = \frac{m_{1}m_{2}(v_{1} + v_{2})^2}{2(m_{1} + m_{2})} $ \\
$Elatisch:  W_{D} = \frac{F_{1} + F_{2}}{2}\Delta s$

\subsubsection{Kernbindungenergie (Einstein)}
$E = m \cdot c^2$

\subsubsection{Leistung}
$P = \frac{dW}{dt} = \frac{Fds}{dt} =\overrightarrow{F} \cdot \overrightarrow{s} = M \cdot \omega$

\subsubsection{Wirkungsgrad}
Beachte: Wo wird die Systemgrenze gezogen?! Wird z.B. die Wärmeleistung weiterverwendet? \\
Wirkungsgrad: $\eta = \frac{P_{ab}}{P_{zu}} = \frac{W_{ab}}{W_{zu}}$ \\
$\eta_{tot} = \eta_{1} \cdot \eta_{2}  \cdot ...$

\subsubsection{Impuls, Impulserhaltung}
Impuls $ \overrightarrow{p} = m \cdot \overrightarrow{v} = \overrightarrow{F}$  \\
Gesamtimpuls: $ p_{ges} = \Sigma_{i = 1} m_{i}v_{s} $\\
Kraftstoss: $\overrightarrow{F} = \frac{\Delta \overrightarrow{p}}{ \Delta t}$

\subsubsection{Drehimpuls}
Drehimpuls: $L = m \cdot v \cdot r \cdot \sin{\phi} = J \cdot \omega$ \\
$\overrightarrow{L} = \overrightarrow{r} \times \overrightarrow{p}$ \\
Drehmoment: $ \overrightarrow{M} = \frac{d \overrightarrow{p}}{dt}
$

\subsubsection{Raketenantrieb}

\subsubsection{Inelastischer Stoss}

\subsubsection{Elastischer Stoss}


